\chapter{Einleitung}

\section{Thema}
Diese Arbeit befasst sich mit der Realisierung einer generalisierten Bibliothek von modernen Algorithmen aus dem Kontext der Schwarmintelligenz in Java, was insbesondere die Untersuchung der drei Algorithmen auf Ähnlichkeiten in der Basis in Hinblick auf mögliche Abstraktionen beinhaltet. 

\section{Zielsetzung}
Als Ziel dieser Arbeit sollen die drei Optimierungsalgorithmen 'Grey Wolf Optimization', 'Rat Swarm Optimization' und 'Elephant Herding Optimization' in einer Java-Bibliothek realisiert werden, um zukünftig in Lehre und Forschung eingesetzt werden zu können.\\
Die Wirksamkeit der Implementierung soll anhand von geeigneten UnitTests anhand der Ackley-Funktion sichergestellt werden.

\section{Abgrenzung}
Diese Arbeit soll keine mathematischen Untersuchungen über das Verständnis der korrespondierenden Paper zu den geforderten Algorithmen hinaus beinhalten und auch keine weiteren Experimente als die Qualitätssicherung anhand der Ackley-Funktion abbilden.

\section{Optimierungsalgorithmen aus dem Bereich der Schwarmintelligenz}
Algorithmen aus dem Bereich der Schwarmintelligenz werden zur Optimierung von Problemen verwendet, in dem Verhaltensstrukturen aus der Natur mathematisch abgebildet und nutzbar gemacht werden.\\
Dabei wird versucht im Verhalten von Lebewesen Muster zu finden, mit denen ein Ziel erreicht werden kann, um somit die Zielfindung mathematischer Probleme zu optimieren. Gesucht wird dabei ein Optimum, also ein globales Minimum oder Maximum einer mehrdimensionalen mathematischen Funktion.\\
Der Algorithmus 'Grey Wolf Optimization' arbeitet rundenbasiert in Iterationen, wobei eine Obergrenze definiert werden kann.

