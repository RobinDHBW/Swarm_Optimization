\chapter{Abstract}

% \section{Deutsch}
% Das Ziel dieses Projektes war der Aufbau eines Software-Stacks, um den Prozess der Temperaturkalibrierung für die Geräteserie der RHEOGRAPHEN automatisieren zu können.\\
% Mit dem Erfolg des Projektes wurde einerseits eine abstrakte Software-Basis aufgebaut, an deren Schnittstellen einfach angedockt werden kann und die, die firmeninterne Software-Plattform PowerTools erweitert. Andererseits wurde der Prozess der Temperaturkalibrierung automatisiert und somit ein weiterer Schritt in Richtung Digitalisierung und papierlose Produktion getätigt.\\
% Es wurden generische Schnittstellen entwickelt, um sowohl \ac{TCP}-basierte, als auch serielle Kommunikation zu ermöglichen. Diese Abstraktionen wurden gleichzeitig genutzt, um spezifische Schnittstellen zu den Geräten der RHEOGRAPHEN-Serie und zu den seriellen Temperaturmessgeräten, die für die Temperaturkalibrierung eingesetzt werden, zu implementieren.
% Hierdurch kann dieser Prozess der Temperaturkalibrierung automatisiert werden und die anfallenden Daten in die bestehenden Strukturen der Datenbank integriert werden, die zur firmeninternen Software PowerTools gehört.\\
% Durch die Automatisierung kann im Produktionsprozess Zeit gespart werden und die gesammelten Daten sind zentralisiert und standardisiert. Diese können somit ausgewertet werden und es können neue Kennzahlen daraus generiert werden, die Aufschluss über den Produktionsprozess und die Historie von produzierten Geräten liefern können.
% \clearpage

% \section{English}
% The target of this project was to develop a software stack, to automate the process of the temperature calibration of the device series RHEOGRAPH.\\
% Within the success of this project an abstract software base was built, that expands the company internal software platform PowerTools with easy usable interfaces. Furthermore the process of the temperature calibration was automated and the digitalization and paperless production inside the company were improved.\\
% Generic interfaces for ether \ac{TCP} based and also serial connections were developed and those abstractions were used to to construct defined interfaces to connect the devices of the RHEOGRAPH series and also to connect the two serial temperature measuring devices, which are used for the temperature calibration process.
% Using this solution the process of temperature calibration could be automated and all collected data could be integrated into already existent database structures dedicated to the software PowerTools.\\
% Throughout this automation time can be saved in the production process and the collected data is saved centralized and stadnardized. So it can be evaluated and new classification numbers can be generated, which can provide information about the production process and the device history of manufactured devices.